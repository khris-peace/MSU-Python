\documentclass[a4paper,12pt]{article}

%\usepackage{cmap}					% поиск в PDF
\usepackage[T2A]{fontenc}			% кодировка
\usepackage[utf8]{inputenc} 	% кодировка исходного текста
\usepackage{mathtools}			
\usepackage[english,russian]{babel}	% локализация и переносы

\author{Lecture 1}
\title{Python}
\date{10/22/2022}


\begin{document} % Конец преамбулы, начало текста.

\maketitle
%write something
Алгоритм -- последовательность команд на формальном языке исполнителя, приводящая к цели за \textbf{конечное время}. 

\textbf{Детерминированность алгоритма!}

Turtle commands:

forward(n) -- продвижение черепашки вперед на n шагов.

backward(n) -- продвижение черепашки назад на n шагов.

right(a) -- поворот на а градусов вправо.

left(a) -- поворот на а градусов влево.

penup() -- поднять "перо"- хвост (после этого все движения черепашки будут оставлять след на экране).

pedown()м -- опустить "перо"- хвост (после этого все движения черепашки будут оставлять след на экране).

speed(m) -- задать скорость m ( m $\in$ [1,13]).

shape("turtle") -- задать форму "черепашка"

goto(x, y) --  перейти в точку с координатами x и y.

\textbf{Цикл for:}

for счетчик in значения, которые нужно перебрать:

4 spaces

Пример цикла для рисования пятиугольника черепашкой:

for step in 1, 2, 3, 4, 5:
    forward(120)
    right(360/5)
print("Это пятиугольник")

Если не поставить 4 пробела перед print, то надпись "Это пятиугольник" будет повторяться в цикле

\textbf{Вложенный цикл}

Цикл в цикле! (см. пример с кодом на рисование пятиугольника, а затем тругольника)

\textbf{Простые функции}

Синтаксис описания простой функции:

def name-of-function():
	
\> \> action1

\> \>action2

\>\>...

Чтобы вызвать функцию надо просто указать ее имя

name-of-function

\textbf{Функции с параметрами:}

def foo(x):

\> \> print('Значение параметра x =' , x)

foo(5)

foo(10)

Значение параметра x = 5

Значение параметра x = 10

При определении функции параметр не определен. Определение параметра происходит при вызове функции

!!!Обратить внимание, что тело внутри цикла изи условия должно отстоять на 4 пробела (1 tab)

По-дефолту функция print() выводит знаечния через запятую. За это отвечает "сепаратор" sep. Если мы хотим изменить сепаратор, то должны указать это в принте:

print(hours, minutes, seconds, sep=':') (example)

Еще один именнованный параметр end=' ' используется для НЕ переноса на новую строку:

print(1,2,3,4,5,sep=':', end=' ')

print(6,7,8,sep='===')

help("print") // ask for help

Рекурсия??? (убрали из программы этого года; можно найти в программе прошлого года)

\end{document}